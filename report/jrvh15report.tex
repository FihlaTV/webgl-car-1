\documentclass[11pt]{article}
\usepackage[utf8]{inputenc}
\usepackage[margin=0.75in]{geometry}
\usepackage{layout}
\usepackage{indentfirst}
\usepackage{url}
\setlength{\voffset}{-0.2in}

%opening
\title{Computer Graphics Summative Assignment}
\author{000280744: jrvh15 - Z0966651}

\begin{document}

\maketitle

\section{Question (a)}
In GLSL (OpenGL Shading Language), qualifiers are given to variables, allowing 
them to hold special meaning. The qualifiers are as follows:
\enumerate
\item{\textbf{Attribute}: Available only in vertex shaders, attribute variables 
are declared as global variables and are used to pass per-vertex data to the 
vertex shader. Variables marked as attribute variables are global 
and may change per vertex; they are used when it is required to set variables 
for every vertex. These variables are passed from the application to the vertex 
shaders. This qualifier can only be used in vertex shaders, and can only be 
read but not written in a vertex shader. This is because they contain vertex 
data, and as such, the values will not be applicable directly in a fragment 
shader.}
\item{\textbf{Uniform}: A uniform variable is a global variable that can have 
its value changed by primitive only, i.e. its value cannot be changed between a 
\texttt{glBegin}/\texttt{glEnd} pair. They are passed into the shaders (can be 
either vertex/fragment shaders - if a uniform variable of the same name and 
data type is declared in both, then the variable is shared between them) from 
the application. This implies that it cannot be used for vertex attributes. They 
are suitable for values that remain constant along a primitive, frame, or even 
the whole scene. Uniform variables are read-only in both vertex and fragment 
shaders.}
\item{\textbf{Varying}: also declared as global, used for interpolated data 
between a vertex and fragment shader by declaring the variable with the same 
type and name in both. In order to compute values per fragment, it is often 
required to access vertex interpolated data. This data may be accessible 
to one but not the other: for example, normals are accessible to the vertex 
shader, but not the fragment shader, being an attribute variable. Thus, once all 
the vertices are processed and at the stage where primitives are assembled and 
fragments computed, for each fragment there is a set of variables that are 
interpolated automatically and provided to the fragment shader. These are 
varying variables, which must be written on a vertex shader, where the value of 
the variable is computed for each vertex. In the fragment shader, this variable 
can only be read.}
\section{Question (b)}

\end{document}
